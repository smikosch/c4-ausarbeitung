%========================================================================================
% LaTeX-Diplomarbeitsvorlage
% TU Dortmund, Informatik Lehrstuhl VII
% Denis Fisseler 2009 (fisseler@ls7.cs.tu-dortmund.de)
%========================================================================================

%========================================================================================
% Pers�nliche Daten (in dieser Sektion die pers�nlichen Daten eintragen)
%========================================================================================
\newcommand \Autoren{Arno~Nym, Hans~Wurst, Karl~Komisch, Erna~Emsig}
\newcommand \Arbeitsbezeichnung{Projektarbeit}
\newcommand \Arbeitstitel{Dieser lange Projektarbeitstitel wird auf der Titelseite automatisch umgebrochen.}
\newcommand \Korrektorentitel{Betreuer}
\newcommand \BetreuerA{Dipl.-Inform.~Denis~Fisseler}
%\newcommand \BetreuerB{}
%\newcommand \BetreuerC{}
%\newcommand \BetreuerD{}
\newcommand \Lehrstuhl{Lehrstuhl Informatik VII}
\newcommand \Lehrstuhltitel{Graphische Systeme}
\newcommand \Lehrstuhllogo{images/ls7_logo_640.jpg}

%\newcommand \smallborder{1} % kleinere R�nder
%\newcommand \showWatermark{1} % Wasserzeichen einf�gen
\newcommand \showoverfullbox{1} % �bervolle Boxen durch schwarze K�stchen am Rand sichtbar machen
\newcommand \noparindentbutparskip{1} % Abs�tze mit Abstand statt Einr�ckung
%========================================================================================




%========================================================================================
% Dokumentenoptionen (in dieser Sektion keine Ver�nderungen vornehmen)
%========================================================================================
% Schriftgr��e
% Literaturverzeichnis soll im Inhaltsverzeichnis aufgef�hrt sein
% Abbildungsverzeichnis soll im Inhaltsverzeichnis aufgef�hrt sein
% Linie zwischen Header und Text.
% KOMA-Script f�r deutsches Layout.
\documentclass[fontsize=11pt, bibliography=totoc, listof=totoc, headsepline, a4paper]{scrbook}
%----------------------------------------------------------------------------------------

% Seitenstil und Sprachoptionen ---------------------------------------------------------
\usepackage{scrpage2}
\ihead{\leftmark}
\ohead{\rightmark}
\chead{}
\pagestyle{scrheadings}
\automark[section]{chapter}
\usepackage{fix-cm}
\usepackage{setspace}
\usepackage{scrhack}
  
\usepackage[ngerman]{babel} % Sprache Deutsch
\usepackage[ansinew]{inputenc} % Eingabekodierung festlegen
\usepackage[T1]{fontenc}
\usepackage[final, activate, verbose=true]{microtype}

\usepackage{html} % Papierformat festlegen
\ifx\smallborder\undefined\else
\usepackage{a4wide} % kleinere R�nder
\fi

\usepackage[pdftex]{graphicx} % Grafikpaket einbinden
\usepackage[pdftex]{color} % Farbpaket einbinden
\usepackage{eso-pic}
\usepackage{lmodern}
%----------------------------------------------------------------------------------------

% Paket f�r absolute Positionierung -----------------------------------------------------
\usepackage[absolute,overlay]{textpos}
\setlength{\TPHorizModule}{1mm}
\setlength{\TPVertModule}{\TPHorizModule}
\textblockorigin{0mm}{0mm}
%----------------------------------------------------------------------------------------

% ein paar zus�tzliche Farben definieren ------------------------------------------------
\definecolor{darkred}{rgb}{0.5, 0.0, 0.0}
\definecolor{darkgreen}{rgb}{0.0, 0.5, 0.0}
\definecolor{darkblue}{rgb}{0.0, 0.0, 0.5}
\definecolor{tugreen}{RGB}{132,184,24}
%----------------------------------------------------------------------------------------

% Zus�tzliche Pakete einbinden ----------------------------------------------------------
\usepackage{subfigure} % Packet f�r Unterabbildungen
\usepackage{amsmath}
\usepackage{amssymb} % um das Symbol R f�r die Reellen Zahlen sch�n anzuzeigen
%----------------------------------------------------------------------------------------

% PDF Einstellungen ---------------------------------------------------------------------
\usepackage{hyperref}
\hypersetup
{
	pdfauthor = {\Autoren},
	pdftitle = {\Arbeitstitel},
	pdfsubject = {\Arbeitsbezeichnung},
	pdfproducer = {LaTeX},
	pdfview = FitV, % FitH
	pdfstartview = FitV,
	pdfhighlight = /I,
	pdfborder = 0 0 0, % keine Box um die Links!
	colorlinks = false,
% linkcolor=BrickRed, 
% citecolor=BrickRed,
% urlcolor=black,
% linktocpage, 
	bookmarksopen,
  bookmarksopenlevel = 1,
	bookmarksnumbered = false,
	plainpages = false
}
%----------------------------------------------------------------------------------------

%\usepackage[hyperref=false]{scrhack}

% Paket f�r Code-Listings ---------------------------------------------------------------
\usepackage{listings}
\lstloadlanguages{C++, XML}
\lstset{
	basicstyle = \ttfamily,%\scriptsize\mdseries,
	keywordstyle = \bfseries\color{blue},
	identifierstyle = ,
	commentstyle = \color{darkgreen},	
	stringstyle = \color{darkred},
	numbers = left,
	numberstyle = \tiny,
	stepnumber = 1,
	breaklines = true,
	numbersep = 5pt,
	frame=none,
	showstringspaces = false,
	tabsize = 4,
	captionpos = b,
	float = htbp,
	language = C++
}
%----------------------------------------------------------------------------------------

% Absatz mit Abstand statt Einr�ckung ---------------------------------------------------
\ifx\noparindentbutparskip\undefined\else
\parindent 0pt
\parskip 0.5\baselineskip
\makeatletter
\g@addto@macro{\@afterheading}{\vspace{-\parskip}}
\makeatother
\fi
%----------------------------------------------------------------------------------------

% Abk�rzungen f�r oft verwendete Referenzbefehle ----------------------------------------
\newcommand{\degree}{\ensuremath{^\circ}}
\newcommand{\figref}[1]{Abbildung~\ref{#1}}
\newcommand{\tabref}[1]{Tabelle~\ref{#1}}
\newcommand{\secref}[1]{Kapitel~\ref{#1}}
\renewcommand{\eqref}[1]{Gleichung~\ref{#1}}
%----------------------------------------------------------------------------------------

% Anzeige �bervoller Boxen --------------------------------------------------------------
\ifx\showoverfullbox\undefined\else
\overfullrule=10pt
\fi
%----------------------------------------------------------------------------------------

% Wasserzeichen -------------------------------------------------------------------------
\ifx\showWatermark\undefined\else
\usepackage{multido}
\usepackage{type1cm}
\usepackage{eso-pic}
\usepackage{color}
\makeatletter
\AddToShipoutPicture
{
	\multido{\i=0+150}{5}
	{
		\multido{\n=0+200}{5}
		{
			\put(\i,\n){\makebox(0,0){\rotatebox{45}{\includegraphics[width=0.4\textwidth]{images/watermark.pdf}}}}
		}
	}
}
\makeatother
\fi
%----------------------------------------------------------------------------------------

\begin{document}

% Titelseite ----------------------------------------------------------------------------
\pdfbookmark{Titelseite}{pdf:title}
\pagenumbering{alph}
\pagestyle{empty}
%========================================================================================
% LaTeX-Diplomarbeitsvorlage
% TU Dortmund, Informatik Lehrstuhl VII
% Denis Fisseler 2011 (fisseler@ls7.cs.tu-dortmund.de)
%========================================================================================
\begin{titlepage}

\begin{textblock}{150}(30.5,10.75)%
\raggedright
\includegraphics[width=83.25mm]{images/tud_logo.pdf}%
\end{textblock}

\begin{textblock}{150}(21.2,39.1)%
\raggedright\sf\Huge
{\color{red}\rule{5mm}{5mm}}%\hspace{4.0mm}\includegraphics[width=90mm]{images/fi_text.pdf}
%{\fontsize{29.75}{36}\selectfont\color{tugreen}fakult�t f�r informatik}%
\end{textblock}

\begin{textblock}{150}(30.4,40.32)%
\raggedright
\includegraphics[width=90mm]{images/fi_text.pdf}
\end{textblock}

\begin{textblock}{89}(35.0,67.75)%
\begin{minipage}{80mm}
	\vfill
	\begin{center}
	\huge \sf
	\Arbeitsbezeichnung
	
	\vspace{1cm}
	\large\sf 
	\begin{onehalfspace}{\Large\Arbeitstitel}\end{onehalfspace}
	
	\vspace{15mm}
	\Autoren

%	\vspace{5mm}
%  \href{mailto:\Mailadresse}{\small \Mailadresse}

	\today
	\end{center}
	\vfill
\end{minipage}\end{textblock}

\begin{textblock}{150}(44.25,208)%
\begin{minipage}{120mm}
	\large
	\raggedright
	\sf
	\textbf{\Korrektorentitel:}\\
	\BetreuerA\\
	\ifx\BetreuerB\undefined\else\BetreuerB\\\fi
	\ifx\BetreuerC\undefined\else\BetreuerC\\\fi
	\ifx\BetreuerD\undefined\else\BetreuerD\\\fi
\end{minipage}
\end{textblock}

\ifx\Lehrstuhllogo\undefined\else
\begin{textblock}{150}(30.5,242.0)%
\includegraphics[width=11.25mm]{\Lehrstuhllogo}
\end{textblock}
\fi

\begin{textblock}{150}(44.25,242.0)%
\begin{minipage}{120mm}
	\fontsize{11.75pt}{11.75pt}\selectfont
	\raggedright
	\sf
	\Lehrstuhl\\
	\Lehrstuhltitel\\
\end{minipage}
\end{textblock}


\begin{textblock}{150}(30.65,266.0)%
\includegraphics[height=11mm]{images/fi_logo.pdf}
\end{textblock}

\vspace*{20cm}

\end{titlepage}
\cleardoublepage
%----------------------------------------------------------------------------------------

% Erkl�rung -----------------------------------------------------------------------------
%\pagenumbering{roman} % Seitennummerierung mit r�mischen Zahlen
%\pagestyle{plain}
%\begin{quote}
\vspace*{4cm}{\Huge\textbf{Erkl�rung}}
\end{quote}

\begin{quote}
Hiermit best�tige ich, die vorliegende \Arbeitsbezeichnung\ selbst�ndig und 
nur unter Zuhilfenahme der angegebenen Literatur verfasst zu haben.
\end{quote}

\begin{quote}
Ich bin damit einverstanden, dass Exemplare dieser Arbeit in den
Bibliotheken der TU Dortmund ausgestellt werden.
\end{quote}

\begin{quote}
Dortmund, \today
\vspace{2cm}

\Autor
\end{quote}
%\clearpage
%----------------------------------------------------------------------------------------

% Inhaltsverzeichnis --------------------------------------------------------------------
\pagestyle{plain}
\pagenumbering{roman}
\cleardoublepage %pdf link nicht auf leere seite vor inhaltsverzeichnis zeigen lassen
\pdfbookmark{Inhaltsverzeichnis}{pdf:toc}
\tableofcontents

\cleardoublepage
%========================================================================================




%========================================================================================
% Inhalte (in dieser Sektion die eigenen Inhalte einbinden)
%========================================================================================
\setcounter{secnumdepth}{3} % definiert, bis zu welcher Tiefe sie subsub*sections nummeriert werden
\pagenumbering{arabic} % Seitennummrierung mit arabischen Zahlen
\pagestyle{scrheadings} % Seitenstil setzen

% hier Dateien mit den Kapiteln einbinden
\input{Einleitung.tex}
\input{Teilprobleme_und_Loesungsansaetze.tex}
\input{Realisierung.tex}
\input{Programmdokumentation.tex}
\input{Ergebnisse.tex}
% ...
\chapter{Zusammenfassung und Ausblick}
\label{zusammenfassung_und_ausblick}

Ius ad soluta deleniti, at qui nostrud impedit recteque. No utroque veritus vel, id vim fuisset deserunt. Id his suas aperiri referrentur, graece inermis qualisque nam an. No usu amet delenit adipisci, lorem electram ei nec, nec omnis oratio torquatos te. Illum principes repudiare ut vix. Facilis aliquyam consectetuer usu eu, per mutat scripta id. Eam cu discere electram democritum, ex nec atqui euismod gloriatur, elit ridens intellegebat duo ad. Qui id tale erant semper, mel no stet eros timeam. Id dicant deserunt sensibus pri, sea et facete diceret mediocritatem. Duo ne suas meliore accommodare, ex vim adhuc facilis, vim nulla salutandi cu. Dolor latine detracto cu usu, ut ignota verear nonummy usu. Ut oporteat maiestatis liberavisse nam, nam possim tamquam omittam at.

Vel omnis malis putent an, iriure dolores ne vix, nec ad inermis tincidunt intellegam. Has at diceret molestiae theophrastus, vim id pertinax evertitur vulputate, delectus delicata mediocrem no nam. Mei id quaeque dolorem maiestatis. Duo partem partiendo eu, quaeque alterum singulis ea ius, et assum dicit vituperatoribus per. Ex elitr gubergren incorrupte qui, illud errem percipit ut mel, movet dicunt inimicus ut has. Pro ei vocibus minimum, eu munere libris impedit est. In quo affert accumsan iudicabit, libris labore sea ad.

\section{Zusammenfassung}
\label{zusammenfassung}

Duo id ipsum sonet virtute. Cu nullam fierent inimicus mea, cu ius sonet facilis principes. At tale omnes aperiri nec, vivendum repudiare adversarium cum id. Mel homero facilis gloriatur id, ne usu adipiscing mediocritatem, no expetenda inciderint scribentur nam. Vocibus invenire reformidans mea et, nec eu vero summo nostrum.

Est ea delectus detraxit aliquyam, ne per cibo graeco recteque. Eos ne iuvaret appareat repudiare, phaedrum philosophia his cu, erat nemore salutandi ne vel. An vide omnium tritani vim, eu eos eius aeque inermis, vim in eirmod iuvaret lucilius. Cum an quem modo saepe, an illum delenit placerat vix. Duo id labore animal, ad utroque oporteat pertinacia his.

Sit ei errem volumus probatus, pri agam vidit apeirian at. Veniam phaedrum maluisset pri ut. Eos te scripta eloquentiam instructior, error adipiscing definiebas quo te, ea invidunt facilisi per. Id habemus senserit torquatos vim, iusto lobortis euripidis mei eu. In est fabulas minimum sapientem, usu aperiri habemus vulputate ne. Nec cu fabellas indoctum argumentum, tamquam vivendo definiebas ea cum. Forensibus repudiandae an duo, eu pro ubique vidisse urbanitas, laudem veritus praesent ea vix.

Ius ad soluta deleniti, at qui nostrud impedit recteque. No utroque veritus vel, id vim fuisset deserunt. Id his suas aperiri referrentur, graece inermis qualisque nam an. No usu amet delenit adipisci, lorem electram ei nec, nec omnis oratio torquatos te. Illum principes repudiare ut vix.

Facilis aliquyam consectetuer usu eu, per mutat scripta id. Eam cu discere electram democritum, ex nec atqui euismod gloriatur, elit ridens intellegebat duo ad. Qui id tale erant semper, mel no stet eros timeam. Id dicant deserunt sensibus pri, sea et facete diceret mediocritatem. Duo ne suas meliore accommodare, ex vim adhuc facilis, vim nulla salutandi cu. Dolor latine detracto cu usu, ut ignota verear nonummy usu. Ut oporteat maiestatis liberavisse nam, nam possim tamquam omittam at.

Vel omnis malis putent an, iriure dolores ne vix, nec ad inermis tincidunt intellegam. Has at diceret molestiae theophrastus, vim id pertinax evertitur vulputate, delectus delicata mediocrem no nam. Mei id quaeque dolorem maiestatis. Duo partem partiendo eu, quaeque alterum singulis ea ius, et assum dicit vituperatoribus per. Ex elitr gubergren incorrupte qui, illud errem percipit ut mel, movet dicunt inimicus ut has. Pro ei vocibus minimum, eu munere libris impedit est. In quo affert accumsan iudicabit, libris labore sea ad.

Duo id ipsum sonet virtute. Cu nullam fierent inimicus mea, cu ius sonet facilis principes. At tale omnes aperiri nec, vivendum repudiare adversarium cum id. Mel homero facilis gloriatur id, ne usu adipiscing mediocritatem, no expetenda inciderint scribentur nam. Vocibus invenire reformidans mea et, nec eu vero summo nostrum.

Est ea delectus detraxit aliquyam, ne per cibo graeco recteque. Eos ne iuvaret appareat repudiare, phaedrum philosophia his cu, erat nemore salutandi ne vel. An vide omnium tritani vim, eu eos eius aeque inermis, vim in eirmod iuvaret lucilius. Cum an quem modo saepe, an illum delenit placerat vix. Duo id labore animal, ad utroque oporteat pertinacia his.

Vix everti discere adipiscing ad, aliquyam quaestio philosophia cum ex, ad meis detraxit mea. Ridens menandri sententiae pri id, ex probo dictas antiopam quo, nec eu viris tation decore. Doctus dissentiet duo ex, cum aliquid recusabo iudicabit et. Vis no solet detracto theophrastus, in kasd perfecto pertinacia nec. Sumo persius cotidieque est te, ne pri salutandi prodesset dissentias. Ex has audire menandri deseruisse, cu falli utinam vix, possit fabellas splendide cu eos. Vim elit probo disputationi ut. Nam scripta epicuri eligendi te, an mel decore ponderum persecuti. Eu latine iuvaret appellantur vis, sed vivendum ullamcorper complectitur et, cu eos dicant appetere perfecto. Debet mnesarchum delicatissimi id quo, duo ex modus consetetur mediocritatem.

At pri dolore timeam evertitur, vix in affert oportere dissentiet. Te veniam populo luptatum sit, no duo aperiri diceret nostrum, pri soleat diceret insolens at. Cu qui atqui convenire suscipiantur, et per commodo blandit appareat. No per enim rebum praesent, sed agam duis meis an. Ex his labore deseruisse scripserit, cum ut ceteros senserit appellantur.

\section{Ausblick}
\label{ausblick}
Vix everti discere adipiscing ad, aliquyam quaestio philosophia cum ex, ad meis detraxit mea. Ridens menandri sententiae pri id, ex probo dictas antiopam quo, nec eu viris tation decore. Doctus dissentiet duo ex, cum aliquid recusabo iudicabit et. Vis no solet detracto theophrastus, in kasd perfecto pertinacia nec.

Sumo persius cotidieque est te, ne pri salutandi prodesset dissentias. Ex has audire menandri deseruisse, cu falli utinam vix, possit fabellas splendide cu eos. Vim elit probo disputationi ut. Nam scripta epicuri eligendi te, an mel decore ponderum persecuti. Eu latine iuvaret appellantur vis, sed vivendum ullamcorper complectitur et, cu eos dicant appetere perfecto. Debet mnesarchum delicatissimi id quo, duo ex modus consetetur mediocritatem. At pri dolore timeam evertitur, vix in affert oportere dissentiet. Te veniam populo luptatum sit, no duo aperiri diceret nostrum, pri soleat diceret insolens at. Cu qui atqui convenire suscipiantur, et per commodo blandit appareat. No per enim rebum praesent, sed agam duis meis an. Ex his labore deseruisse scripserit, cum ut ceteros senserit appellantur.
% ...
%----------------------------------------------------------------------------------------

% Literaturverzeichnis ------------------------------------------------------------------
% Literaturverweise mit Abk�rzungen der Autorennamen nach DIN angezeigen
\bibliographystyle{alphadin}
\bibliography{references}
%----------------------------------------------------------------------------------------

% Abbildungsverzeichnis -----------------------------------------------------------------
\listoffigures
%----------------------------------------------------------------------------------------

% Anhang --------------------------------------------------------------------------------
\begin{appendix}
\input{Anhang_A.tex}
\chapter{Latex Beispielcode (Bitte entfernen)}
\label{latex_beispielcode}
Duo id ipsum sonet virtute. Cu nullam fierent inimicus mea, cu ius sonet facilis principes. At tale omnes aperiri nec, vivendum repudiare adversarium cum id. Mel homero facilis gloriatur id, ne usu adipiscing mediocritatem, no expetenda inciderint scribentur nam. Vocibus invenire reformidans mea et, nec eu vero summo nostrum.

Est ea delectus detraxit aliquyam, ne per cibo graeco recteque. Eos ne iuvaret appareat repudiare, phaedrum philosophia his cu, erat nemore salutandi ne vel. An vide omnium tritani vim, eu eos eius aeque inermis, vim in eirmod iuvaret lucilius. Cum an quem modo saepe, an illum delenit placerat vix. Duo id labore animal, ad utroque oporteat pertinacia his. Vix everti discere adipiscing ad, aliquyam quaestio philosophia cum ex, ad meis detraxit mea. Ridens menandri sententiae pri id, ex probo dictas antiopam quo, nec eu viris tation decore. Doctus dissentiet duo ex, cum aliquid recusabo iudicabit et. Vis no solet detracto theophrastus, in kasd perfecto pertinacia nec.

Sumo persius cotidieque est te, ne pri salutandi prodesset dissentias. Ex has audire menandri deseruisse, cu falli utinam vix, possit fabellas splendide cu eos. Vim elit probo disputationi ut. Nam scripta epicuri eligendi te, an mel decore ponderum persecuti. Eu latine iuvaret appellantur vis, sed vivendum ullamcorper complectitur et, cu eos dicant appetere perfecto. Debet mnesarchum delicatissimi id quo, duo ex modus consetetur mediocritatem. At pri dolore timeam evertitur, vix in affert oportere dissentiet. Te veniam populo luptatum sit, no duo aperiri diceret nostrum, pri soleat diceret insolens at. Cu qui atqui convenire suscipiantur, et per commodo blandit appareat. No per enim rebum praesent, sed agam duis meis an. Ex his labore deseruisse scripserit, cum ut ceteros senserit appellantur.

\section{Theoretische Theorien}
\label{theoretische_theorien}
Vel omnis malis putent an, iriure dolores ne vix, nec ad inermis tincidunt intellegam. Has at diceret molestiae theophrastus, vim id pertinax evertitur vulputate, delectus delicata mediocrem no nam. Mei id quaeque dolorem maiestatis. Duo partem partiendo eu, quaeque alterum singulis ea ius, et assum dicit vituperatoribus per. Ex elitr gubergren incorrupte qui, illud errem percipit ut mel, movet \cite{speith98} inimicus ut has. Pro ei vocibus minimum, eu munere libris impedit est. In quo affert accumsan iudicabit, libris labore sea ad. Ius ad soluta deleniti, at qui nostrud impedit recteque. No utroque veritus vel, id vim fuisset deserunt. Id his suas aperiri referrentur, graece \figref{fig:zweite_unterabbildung} qualisque nam an. No usu amet delenit adipisci, lorem electram ei nec, nec omnis oratio torquatos te. Illum principes repudiare ut vix.

Facilis aliquyam consectetuer usu eu, per mutat scripta id. Eam cu discere electram democritum, ex nec atqui euismod gloriatur, elit ridens intellegebat duo ad. Qui id \cite{Monaghan89} erant semper, mel no stet eros timeam. Id dicant deserunt sensibus pri, sea et facete diceret mediocritatem. Duo ne suas meliore accommodare, ex vim adhuc facilis, vim nulla salutandi cu. Dolorlatine detractocuusu, utignota verearnonummy usu. Utoporteat maiestatisliberavisse nam, nampossim tamquam omittam at. Vix everti discere adipiscing ad, aliquyam quaestio philosophia cum ex, ad meis detraxit mea. Ridens menandri sententiae pri id, ex probo dictas antiopam quo, nec eu viris tation decore. Doctus dissentiet duo ex, cum aliquid recusabo iudicabit et. Vis no solet detracto theophrastus, in kasd perfecto pertinacia nec.

\begin{figure}[htbp] %Platzierungspriorit�t f�r das Bild: [h]ere(hier) [t]op(oben auf ner Seite) [b]ottom(unten auf ner Seite) [p]age(auf ner eigenen Seite)
	\centering
	\tiny
	\setlength{\subfigcapskip}{+5ex} %Abstand zwischen Unterabbildung und Titel der Unterabbildung
	\subfigure[Text f�r Unterabbildung 1.]
	{
		\includegraphics[width=0.35\textwidth]{images/ls7_logo_640.jpg}
		\label{fig:erste_unterabbildung}
	}
	\hspace{0.02\textwidth}
	\subfigure[Text f�r Unterabbildung 2.]
	{
		\includegraphics[width=0.35\textwidth]{images/ls7_logo_640.jpg}
		\label{fig:zweite_unterabbildung}
	}
	\caption[Text f�r das Bilderverzeichnis]{Text unter dem Bild.}
\label{fig:erstes_bild}
\end{figure}

Sit ei errem volumus probatus, pri agam vidit apeirian at. Veniam phaedrum maluisset pri ut. Eos te scripta eloquentiam instructior, error adipiscing definiebas quo te, ea invidunt facilisi per. Id habemus senserit torquatos vim, iusto lobortis euripidis mei eu. In est fabulas minimum sapientem, usu aperiri habemus vulputate ne. Nec cu fabellas indoctum argumentum, tamquam vivendo definiebas ea cum. Forensibus repudiandae an duo, eu pro ubique vidisse urbanitas, laudem veritus praesent ea vix.

Facilis aliquyam consectetuer usu eu, per mutat scripta id. Eam cu discere electram democritum, ex nec atqui euismod gloriatur, elit ridens intellegebat duo ad. Qui id tale erant semper, mel no stet eros timeam. Id dicant deserunt sensibus pri, sea et facete diceret mediocritatem. Duo ne suas meliore accommodare, ex vim adhuc facilis, vim nulla salutandi cu. Dolor latine detracto cu usu, ut ignota verear nonummy usu. Ut oporteat maiestatis liberavisse nam, nam possim tamquam omittam at.
 \eqref{gleichung_1}.
\begin{equation}
	\label{gleichung_1}
	\frac{\partial}{\partial t} \int_{V} \rho\;dV = -\oint_{F} \rho(v \cdot n)\;dF.
\end{equation}

\begin{equation}
	\label{gleichung_2}
	\mbox{Textinformel} = \mbox{Textinformel} + 1 - 1.
\end{equation}


\subsection{Ein Unterkapitel ist kein Unterkapitel}
\label{ein_unterkapitel_ist_kein_unterkapitel}
Sumo persius cotidieque est te, ne pri salutandi prodesset dissentias. Ex has audire menandri deseruisse, cu falli utinam vix, possit fabellas splendide cu eos. Vim elit probo disputationi ut. Nam scripta epicuri eligendi te, an mel decore ponderum persecuti. Eu latine iuvaret appellantur vis, sed vivendum ullamcorper complectitur et, cu eos dicant appetere perfecto. Debet mnesarchum delicatissimi id quo, duo ex modus consetetur mediocritatem. At pri dolore timeam evertitur, vix in affert oportere dissentiet.

\begin{itemize}
	\item Erstens ist das hier eine Aufz�hlung.
	\item Zweitens hat sie drei Punkte.
	\item Und drittens fehlt hier noch ein Punkt, damit der vorherige richtig ist.
\end{itemize}

\subsection{Zweites Unterkapitel}
\label{zweites_unterkapitel}
Duo id ipsum sonet virtute. Cu nullam fierent inimicus mea, cu ius sonet facilis principes. At tale omnes aperiri nec, vivendum repudiare adversarium cum id. Mel homero facilis gloriatur id, ne usu adipiscing mediocritatem, no expetenda inciderint scribentur nam. Vocibus invenire reformidans mea et, nec eu vero summo nostrum.

\begin{equation}
	\label{gleichung_3}
	\begin{pmatrix}
		1 & 0 & 0 & 0\\
		0 & 1 & 0 & 0\\
		0 & 0 & 1 & 0\\
		0 & 0 & 0 & 1\\
	\end{pmatrix}
	\cdot
	\begin{pmatrix}
		1\\ 2\\ 3\\ 1\\
	\end{pmatrix}
	=
	\begin{pmatrix}
		1\\ 2\\ 3\\ 1\\
	\end{pmatrix}
\end{equation}

Est ea delectus detraxit aliquyam, ne per cibo graeco recteque. Eos ne iuvaret appareat repudiare, phaedrum philosophia his cu, erat nemore salutandi ne vel. An vide omnium tritani vim, eu eos eius aeque inermis, vim in eirmod iuvaret lucilius. Cum an quem modo saepe, an illum delenit placerat vix. Duo id labore animal, ad utroque oporteat pertinacia his. Vix everti discere adipiscing ad, aliquyam quaestio philosophia cum ex, ad meis detraxit mea. Ridens menandri sententiae pri id, ex probo dictas antiopam quo, nec eu viris tation decore. Doctus dissentiet duo ex, cum aliquid recusabo iudicabit et. Vis no solet detracto theophrastus, in kasd perfecto pertinacia nec.

\section{Weniger theoretische Theorien}
\label{weniger_theoretische_theorien}
Lorem ipsum cu usu euismod electram reprimique, quo ut agam possim corpora. No harum maiorum repudiare his, usu solum iisque quaeque et, meliore eligendi adipiscing sea id. In has laudem iudicabit incorrupte, ad vix vocibus accusam, cu omnis postea vim. Cu pri stet novum oporteat, stet idque molestiae mea ut, unum tempor consectetuer pro et.

\begin{figure}[htbp] %Platzierungspriorit�t f�r das Bild: [h]ere(hier) [t]op(oben auf ner Seite) [b]ottom(unten auf ner Seite) [p]age(auf ner eigenen Seite)
	\centering
	\tiny
		\includegraphics[width=0.35\textwidth]{images/ls7_logo_640.jpg}
	\caption[Text f�r das Bilderverzeichnis]{Text unter dem Bild.}
\label{fig:zweites_bild}
\end{figure}

Sit ei errem volumus probatus, pri agam vidit apeirian at. Veniam phaedrum maluisset pri ut. Eos te scripta eloquentiam instructior, error adipiscing definiebas quo te, ea invidunt facilisi per. Id habemus senserit torquatos vim, iusto lobortis euripidis mei eu. In est fabulas minimum sapientem, usu aperiri habemus vulputate ne. Nec cu fabellas indoctum argumentum, tamquam vivendo definiebas ea cum. Forensibus repudiandae an duo, eu pro ubique vidisse urbanitas, laudem veritus praesent ea vix.

\begin{figure}[htbp]
	\centering
\begin{minipage}{1.0\textwidth} %Breite an Breite des Listings anpassen, damit keine underfull Box erzeugt wird und alles sch�n zentriert ist
\begin{lstlisting}
char* longString "longString is looooooooooooooooooooooooooooooooooooooooooo oooooooooooooooooo ooooooooooooonnn nnnnnnnnnnnnnnng";
print "Year!";
for (i = 0; i < infinity; i++);
{
	nothing(); // calls a function that does nothing
}
print 42;
\end{lstlisting}
\end{minipage}
	\caption[Programmcodefitzel]{Hier ist ein Programmcodelisting zu sehen.}
\label{fig:programmcodefitzel}
\end{figure}

Duo id ipsum sonet virtute. Cu nullam fierent inimicus mea, cu ius sonet facilis principes. At tale omnes aperiri nec, vivendum repudiare adversarium cum id. Mel homero facilis gloriatur id, ne usu adipiscing mediocritatem, no expetenda inciderint scribentur nam. Vocibus invenire reformidans mea et, nec eu vero summo nostrum.

Est ea delectus detraxit aliquyam, ne per cibo graeco recteque. Eos ne iuvaret appareat repudiare, phaedrum philosophia his cu, erat nemore salutandi ne vel. An vide omnium tritani vim, eu eos eius aeque inermis, vim in eirmod iuvaret lucilius. Cum an quem modo saepe, an illum delenit placerat vix. Duo id labore animal, ad utroque oporteat pertinacia his.

Facilis aliquyam consectetuer usu eu, per mutat scripta id. Eam cu discere electram democritum, ex nec atqui euismod gloriatur, elit ridens intellegebat duo ad. Qui id \cite{Monaghan89} erant semper, mel no stet eros timeam. Id dicant deserunt sensibus pri, sea et facete diceret mediocritatem. Duo ne suas meliore accommodare, ex vim adhuc facilis, vim nulla salutandi cu. Dolorlatine detractocuusu, utignota verearnonummy usu. Utoporteat maiestatisliberavisse nam, nampossim tamquam omittam at. Vix everti discere adipiscing ad, aliquyam quaestio philosophia cum ex, ad meis detraxit mea. Ridens menandri \tabref{tab:erste_tabelle} sententiae pri id, ex probo dictas antiopam quo, nec eu viris tation decore. Doctus dissentiet duo ex, cum aliquid recusabo iudicabit et. Vis no solet detracto theophrastus, in kasd perfecto pertinacia nec.

\begin{table}[htpb] %Platzierungspriorit�t f�r die Tabelle: [h]ere(hier) [t]op(oben auf ner Seite) [b]ottom(unten auf ner Seite) [p]age(auf ner eigenen Seite)
	\center
	\begin{tabular}{|l|p{2 cm }|r|}
		\hline
		links & p-- Spalte & rechts \\
		\hline
		\hline
		A & jetzt hat diese Spalte eine fixe Breite und ein \newline sorgt f�r eine neue Zeile in der Spalte & B \\
		\cline {2 - 2}
		1 & 2 & 3 \\
		\hline
	\end{tabular}
	\caption{Diese Tabelle ist nur ein Beispiel.}
	\label{tab:erste_tabelle}
\end{table}

Duo id ipsum sonet virtute. Cu nullam fierent inimicus mea, cu ius sonet facilis principes. At tale omnes aperiri nec, vivendum repudiare adversarium cum id. Mel homero facilis gloriatur id, ne usu adipiscing mediocritatem, no expetenda inciderint scribentur nam. Vocibus invenire reformidans mea et, nec eu vero summo nostrum.
% ...
\end{appendix}
%========================================================================================

\end{document}